%% Generated by Sphinx.
\def\sphinxdocclass{report}
\documentclass[letterpaper,10pt,russian]{sphinxmanual}
\ifdefined\pdfpxdimen
   \let\sphinxpxdimen\pdfpxdimen\else\newdimen\sphinxpxdimen
\fi \sphinxpxdimen=.75bp\relax

\PassOptionsToPackage{warn}{textcomp}
\usepackage[utf8]{inputenc}
\ifdefined\DeclareUnicodeCharacter
% support both utf8 and utf8x syntaxes
  \ifdefined\DeclareUnicodeCharacterAsOptional
    \def\sphinxDUC#1{\DeclareUnicodeCharacter{"#1}}
  \else
    \let\sphinxDUC\DeclareUnicodeCharacter
  \fi
  \sphinxDUC{00A0}{\nobreakspace}
  \sphinxDUC{2500}{\sphinxunichar{2500}}
  \sphinxDUC{2502}{\sphinxunichar{2502}}
  \sphinxDUC{2514}{\sphinxunichar{2514}}
  \sphinxDUC{251C}{\sphinxunichar{251C}}
  \sphinxDUC{2572}{\textbackslash}
\fi
\usepackage{cmap}
\usepackage[T1]{fontenc}
\usepackage{amsmath,amssymb,amstext}
\usepackage{babel}




\usepackage[Sonny]{fncychap}
\ChNameVar{\Large\normalfont\sffamily}
\ChTitleVar{\Large\normalfont\sffamily}
\usepackage{sphinx}

\fvset{fontsize=\small}
\usepackage{geometry}


% Include hyperref last.
\usepackage{hyperref}
% Fix anchor placement for figures with captions.
\usepackage{hypcap}% it must be loaded after hyperref.
% Set up styles of URL: it should be placed after hyperref.
\urlstyle{same}
\addto\captionsrussian{\renewcommand{\contentsname}{Contents:}}

\usepackage{sphinxmessages}
\setcounter{tocdepth}{1}



\title{OCRLibrary}
\date{дек. 22, 2019}
\release{0.0.1}
\author{A.\@{} Andreev, V.\@{} Loik, E.\@{} Myasnikova, K.\@{} Sharov}
\newcommand{\sphinxlogo}{\vbox{}}
\renewcommand{\releasename}{Выпуск}
\makeindex
\begin{document}

\ifdefined\shorthandoff
  \ifnum\catcode`\=\string=\active\shorthandoff{=}\fi
  \ifnum\catcode`\"=\active\shorthandoff{"}\fi
\fi

\pagestyle{empty}
\sphinxmaketitle
\pagestyle{plain}
\sphinxtableofcontents
\pagestyle{normal}
\phantomsection\label{\detokenize{index::doc}}



\chapter{Pre install requirements}
\label{\detokenize{Installation:pre-install-requirements}}\label{\detokenize{Installation::doc}}
Перед использованием библиотеки OCR необходимо установить следующие пакеты языка программирования Python:
\begin{quote}
\begin{itemize}
\item {} 
NumPy версии 1.17.2 или выше

\end{itemize}

\begin{sphinxVerbatim}[commandchars=\\\{\}]
\PYG{n}{pip} \PYG{n}{install} \PYG{l+s+s1}{\PYGZsq{}}\PYG{l+s+s1}{numpy\PYGZgt{}=1.17.2}\PYG{l+s+s1}{\PYGZsq{}}
\end{sphinxVerbatim}
\begin{itemize}
\item {} 
Pillow версии 6.1.0 или выше

\end{itemize}

\begin{sphinxVerbatim}[commandchars=\\\{\}]
\PYG{n}{pip} \PYG{n}{install} \PYG{l+s+s1}{\PYGZsq{}}\PYG{l+s+s1}{Pillow\PYGZgt{}=6.1.0}\PYG{l+s+s1}{\PYGZsq{}}
\end{sphinxVerbatim}
\begin{itemize}
\item {} 
tqdm версии 4.40.2 или выше

\end{itemize}

\begin{sphinxVerbatim}[commandchars=\\\{\}]
\PYG{n}{pip} \PYG{n}{install} \PYG{l+s+s1}{\PYGZsq{}}\PYG{l+s+s1}{tqdm\PYGZgt{}=4.40.2}\PYG{l+s+s1}{\PYGZsq{}}
\end{sphinxVerbatim}
\end{quote}


\chapter{Example of using library}
\label{\detokenize{UsingExample:example-of-using-library}}\label{\detokenize{UsingExample::doc}}
Необходимо импортировать класс {\hyperref[\detokenize{OCRImage:image.OCRImage}]{\sphinxcrossref{\sphinxcode{\sphinxupquote{OCRImage}}}}}, инициализировать его экземпляр и использовать необходимые вам функции.

Пример использования библиотеки:

\begin{sphinxVerbatim}[commandchars=\\\{\}]
\PYG{k+kn}{from} \PYG{n+nn}{image} \PYG{k+kn}{import} \PYG{n}{OCRImage}

\PYG{n}{img} \PYG{o}{=} \PYG{n}{OCRImage}\PYG{p}{(}\PYG{l+s+s2}{\PYGZdq{}}\PYG{l+s+s2}{path/to/image.bmp}\PYG{l+s+s2}{\PYGZdq{}}\PYG{p}{)}
\PYG{n}{img}\PYG{o}{.}\PYG{n}{binary\PYGZus{}image\PYGZus{}object}\PYG{o}{.}\PYG{n}{cristian\PYGZus{}binarisation}\PYG{p}{(}\PYG{p}{)}
\PYG{n}{img}\PYG{o}{.}\PYG{n}{show\PYGZus{}result}\PYG{p}{(}\PYG{p}{)}
\end{sphinxVerbatim}


\chapter{OCRImage}
\label{\detokenize{OCRImage:ocrimage}}\label{\detokenize{OCRImage::doc}}\index{OCRImage (класс в image)@\spxentry{OCRImage}\spxextra{класс в image}}

\begin{fulllineitems}
\phantomsection\label{\detokenize{OCRImage:image.OCRImage}}\pysiglinewithargsret{\sphinxbfcode{\sphinxupquote{class }}\sphinxcode{\sphinxupquote{image.}}\sphinxbfcode{\sphinxupquote{OCRImage}}}{\emph{path}}{}~\begin{description}
\item[{Класс предоставляющий функционал для обработки изображения следующими способами:}] \leavevmode\begin{itemize}
\item {} 
\sphinxcode{\sphinxupquote{get\_grayscale\_image}} \textendash{} получение монохромного изображения

\item {} 
\sphinxcode{\sphinxupquote{get\_binary\_image}} \textendash{} получение бинаризованного изображения

\item {} 
\sphinxcode{\sphinxupquote{get\_filtered\_image}} \textendash{} получение отфильтрованного изображения

\item {} 
\sphinxcode{\sphinxupquote{get\_contoured\_image}} \textendash{} получение контурного изображения

\item {} 
\sphinxcode{\sphinxupquote{get\_text\_profiled\_image}} \textendash{} выделение символов в текстовом изображения

\item {} 
\sphinxcode{\sphinxupquote{get\_text\_recognized\_image}} \textendash{} распознавание текста на изображении

\end{itemize}

\end{description}
\index{get\_binary\_image() (метод image.OCRImage)@\spxentry{get\_binary\_image()}\spxextra{метод image.OCRImage}}

\begin{fulllineitems}
\phantomsection\label{\detokenize{OCRImage:image.OCRImage.get_binary_image}}\pysiglinewithargsret{\sphinxbfcode{\sphinxupquote{get\_binary\_image}}}{\emph{method=None}, \emph{\_rsize=3}, \emph{\_Rsize=15}, \emph{\_eps=15}, \emph{\_w\_size=15}, \emph{\_k=0.5}}{}~\begin{description}
\item[{Возвращает изображение бинаризованное выбранным методом:}] \leavevmode\begin{itemize}
\item {} 
1 \sphinxhyphen{} для метода Эйквила

\item {} 
2 \sphinxhyphen{} для метода Кристиана

\end{itemize}

\end{description}
\begin{quote}\begin{description}
\item[{Параметры}] \leavevmode\begin{itemize}
\item {} 
\sphinxstyleliteralstrong{\sphinxupquote{method}} (\sphinxstyleliteralemphasis{\sphinxupquote{int}}\sphinxstyleliteralemphasis{\sphinxupquote{ or }}\sphinxstyleliteralemphasis{\sphinxupquote{None}}) \textendash{} метод бинаризации

\item {} 
\sphinxstyleliteralstrong{\sphinxupquote{\_rsize}} (\sphinxstyleliteralemphasis{\sphinxupquote{int}}) \textendash{} размер малого окна

\item {} 
\sphinxstyleliteralstrong{\sphinxupquote{\_Rsize}} (\sphinxstyleliteralemphasis{\sphinxupquote{int}}) \textendash{} размер большего окна

\item {} 
\sphinxstyleliteralstrong{\sphinxupquote{\_eps}} (\sphinxstyleliteralemphasis{\sphinxupquote{int}}) \textendash{} величина отклонения для математических ожиданий чёрного и белого, в пределах которого можно считать         , что они отличются несущественно

\item {} 
\sphinxstyleliteralstrong{\sphinxupquote{\_w\_size}} (\sphinxstyleliteralemphasis{\sphinxupquote{inr}}) \textendash{} размер окна

\item {} 
\sphinxstyleliteralstrong{\sphinxupquote{\_k}} (\sphinxstyleliteralemphasis{\sphinxupquote{float}}) \textendash{} коэффициент, отвечающий за чувствительность бинаризатора

\end{itemize}

\item[{Результат}] \leavevmode
\sphinxcode{\sphinxupquote{Image}} \textendash{} бинаризованное изображение

\item[{Raises}] \leavevmode
ValueError

\end{description}\end{quote}

\end{fulllineitems}

\index{get\_contoured\_image() (метод image.OCRImage)@\spxentry{get\_contoured\_image()}\spxextra{метод image.OCRImage}}

\begin{fulllineitems}
\phantomsection\label{\detokenize{OCRImage:image.OCRImage.get_contoured_image}}\pysiglinewithargsret{\sphinxbfcode{\sphinxupquote{get\_contoured\_image}}}{\emph{method=None}, \emph{t=None}}{}~\begin{description}
\item[{Возвращает контурное изображение вычисленное выбранным методом:}] \leavevmode\begin{itemize}
\item {} 
1 \sphinxhyphen{} для оператора Собеля

\item {} 
2 \sphinxhyphen{} для оператора Прюита

\end{itemize}

\end{description}
\begin{quote}\begin{description}
\item[{Параметры}] \leavevmode\begin{itemize}
\item {} 
\sphinxstyleliteralstrong{\sphinxupquote{method}} (\sphinxstyleliteralemphasis{\sphinxupquote{int}}\sphinxstyleliteralemphasis{\sphinxupquote{ or }}\sphinxstyleliteralemphasis{\sphinxupquote{None}}) \textendash{} метод бинаризации

\item {} 
\sphinxstyleliteralstrong{\sphinxupquote{t}} (\sphinxstyleliteralemphasis{\sphinxupquote{int}}) \textendash{} порог

\end{itemize}

\item[{Результат}] \leavevmode
\sphinxcode{\sphinxupquote{Image}} \textendash{} бинаризованное изображение

\item[{Raises}] \leavevmode
ValueError

\end{description}\end{quote}

\end{fulllineitems}

\index{get\_filtered\_image() (метод image.OCRImage)@\spxentry{get\_filtered\_image()}\spxextra{метод image.OCRImage}}

\begin{fulllineitems}
\phantomsection\label{\detokenize{OCRImage:image.OCRImage.get_filtered_image}}\pysiglinewithargsret{\sphinxbfcode{\sphinxupquote{get\_filtered\_image}}}{\emph{method=None}, \emph{rank=None}, \emph{wsize=3}}{}~\begin{description}
\item[{Возвращает изображение отфильтрованное выбранным методом:}] \leavevmode\begin{itemize}
\item {} 
1 \sphinxhyphen{} для медианного фильтра

\item {} 
2 \sphinxhyphen{} для взвешенного рангового фильтра

\item {} 
3 \sphinxhyphen{} для раногового фильтра

\end{itemize}

\end{description}
\begin{quote}\begin{description}
\item[{Параметры}] \leavevmode\begin{itemize}
\item {} 
\sphinxstyleliteralstrong{\sphinxupquote{method}} (\sphinxstyleliteralemphasis{\sphinxupquote{int}}\sphinxstyleliteralemphasis{\sphinxupquote{ or }}\sphinxstyleliteralemphasis{\sphinxupquote{None}}) \textendash{} метод фильтрации

\item {} 
\sphinxstyleliteralstrong{\sphinxupquote{rank}} (\sphinxstyleliteralemphasis{\sphinxupquote{int}}) \textendash{} ранг фильтра

\item {} 
\sphinxstyleliteralstrong{\sphinxupquote{wsize}} (\sphinxstyleliteralemphasis{\sphinxupquote{int}}) \textendash{} размер окна фильтрации

\end{itemize}

\item[{Результат}] \leavevmode
\sphinxcode{\sphinxupquote{Image}} \textendash{} бинаризованное изображение

\item[{Raises}] \leavevmode
ValueError

\end{description}\end{quote}

\end{fulllineitems}

\index{get\_grayscale\_image() (метод image.OCRImage)@\spxentry{get\_grayscale\_image()}\spxextra{метод image.OCRImage}}

\begin{fulllineitems}
\phantomsection\label{\detokenize{OCRImage:image.OCRImage.get_grayscale_image}}\pysiglinewithargsret{\sphinxbfcode{\sphinxupquote{get\_grayscale\_image}}}{}{}
Возвращает монохромное изображение

\end{fulllineitems}

\index{get\_text\_profiled\_image() (метод image.OCRImage)@\spxentry{get\_text\_profiled\_image()}\spxextra{метод image.OCRImage}}

\begin{fulllineitems}
\phantomsection\label{\detokenize{OCRImage:image.OCRImage.get_text_profiled_image}}\pysiglinewithargsret{\sphinxbfcode{\sphinxupquote{get\_text\_profiled\_image}}}{\emph{text=\textquotesingle{}Привет мир\textquotesingle{}}, \emph{font\_size=36}, \emph{font=\textquotesingle{}TNR.ttf\textquotesingle{}}, \emph{image\_size=(600}, \emph{600)}, \emph{filename=\textquotesingle{}text\textquotesingle{}}}{}
Возвращает изображение с выделенными сегментами символов на сгенерированном изображении теста
\begin{quote}\begin{description}
\item[{Параметры}] \leavevmode\begin{itemize}
\item {} 
\sphinxstyleliteralstrong{\sphinxupquote{text}} (\sphinxstyleliteralemphasis{\sphinxupquote{str}}\sphinxstyleliteralemphasis{\sphinxupquote{ or }}\sphinxstyleliteralemphasis{\sphinxupquote{None}}) \textendash{} текст

\item {} 
\sphinxstyleliteralstrong{\sphinxupquote{font\_size}} (\sphinxstyleliteralemphasis{\sphinxupquote{int}}) \textendash{} размер шрифта

\item {} 
\sphinxstyleliteralstrong{\sphinxupquote{font}} (\sphinxstyleliteralemphasis{\sphinxupquote{str}}\sphinxstyleliteralemphasis{\sphinxupquote{ or }}\sphinxstyleliteralemphasis{\sphinxupquote{None}}) \textendash{} путь до файла шрифта

\item {} 
\sphinxstyleliteralstrong{\sphinxupquote{image\_size}} (\sphinxstyleliteralemphasis{\sphinxupquote{tuple}}) \textendash{} размер символа

\item {} 
\sphinxstyleliteralstrong{\sphinxupquote{filename}} (\sphinxstyleliteralemphasis{\sphinxupquote{str}}\sphinxstyleliteralemphasis{\sphinxupquote{ or }}\sphinxstyleliteralemphasis{\sphinxupquote{None}}) \textendash{} путь до файла сгенерированного текста

\end{itemize}

\item[{Результат}] \leavevmode
\sphinxcode{\sphinxupquote{Image}} \textendash{} бинаризованное изображение

\item[{Raises}] \leavevmode
ValueError

\end{description}\end{quote}

\end{fulllineitems}

\index{get\_text\_recognized\_image() (метод image.OCRImage)@\spxentry{get\_text\_recognized\_image()}\spxextra{метод image.OCRImage}}

\begin{fulllineitems}
\phantomsection\label{\detokenize{OCRImage:image.OCRImage.get_text_recognized_image}}\pysiglinewithargsret{\sphinxbfcode{\sphinxupquote{get\_text\_recognized\_image}}}{\emph{text=\textquotesingle{}Привет мир\textquotesingle{}}, \emph{font\_size=36}, \emph{font=\textquotesingle{}TNR.ttf\textquotesingle{}}, \emph{image\_size=(600}, \emph{600)}, \emph{filename=\textquotesingle{}text\textquotesingle{}}}{}
Возвращает распознаный на сгенерированном изображении текст
\begin{quote}\begin{description}
\item[{Параметры}] \leavevmode\begin{itemize}
\item {} 
\sphinxstyleliteralstrong{\sphinxupquote{text}} (\sphinxstyleliteralemphasis{\sphinxupquote{str}}\sphinxstyleliteralemphasis{\sphinxupquote{ or }}\sphinxstyleliteralemphasis{\sphinxupquote{None}}) \textendash{} текст

\item {} 
\sphinxstyleliteralstrong{\sphinxupquote{font\_size}} (\sphinxstyleliteralemphasis{\sphinxupquote{int}}) \textendash{} размер шрифта

\item {} 
\sphinxstyleliteralstrong{\sphinxupquote{font}} (\sphinxstyleliteralemphasis{\sphinxupquote{str}}\sphinxstyleliteralemphasis{\sphinxupquote{ or }}\sphinxstyleliteralemphasis{\sphinxupquote{None}}) \textendash{} путь до файла шрифта

\item {} 
\sphinxstyleliteralstrong{\sphinxupquote{image\_size}} (\sphinxstyleliteralemphasis{\sphinxupquote{tuple}}) \textendash{} размер символа

\item {} 
\sphinxstyleliteralstrong{\sphinxupquote{filename}} (\sphinxstyleliteralemphasis{\sphinxupquote{str}}\sphinxstyleliteralemphasis{\sphinxupquote{ or }}\sphinxstyleliteralemphasis{\sphinxupquote{None}}) \textendash{} путь до файла сгенерированного текста

\end{itemize}

\item[{Результат}] \leavevmode
str \textendash{} распознаный текст

\item[{Raises}] \leavevmode
ValueError

\end{description}\end{quote}

\end{fulllineitems}

\index{save() (метод image.OCRImage)@\spxentry{save()}\spxextra{метод image.OCRImage}}

\begin{fulllineitems}
\phantomsection\label{\detokenize{OCRImage:image.OCRImage.save}}\pysiglinewithargsret{\sphinxbfcode{\sphinxupquote{save}}}{\emph{path: str}}{}
Сохраняет изображение обработанное в виде BMP изображения под заданным в path путём к файлу
\begin{quote}\begin{description}
\item[{Параметры}] \leavevmode
\sphinxstyleliteralstrong{\sphinxupquote{path}} (\sphinxstyleliteralemphasis{\sphinxupquote{str}}) \textendash{} путь к файлу для сохранения

\end{description}\end{quote}

\end{fulllineitems}

\index{show\_result() (метод image.OCRImage)@\spxentry{show\_result()}\spxextra{метод image.OCRImage}}

\begin{fulllineitems}
\phantomsection\label{\detokenize{OCRImage:image.OCRImage.show_result}}\pysiglinewithargsret{\sphinxbfcode{\sphinxupquote{show\_result}}}{}{}
Отображает результат последнего преобразования изображения,
при отсутствии такового отображает исходное изображение

\end{fulllineitems}


\end{fulllineitems}



\chapter{Built\sphinxhyphen{}in Modules}
\label{\detokenize{BuiltIn:built-in-modules}}\label{\detokenize{BuiltIn::doc}}

\section{LabImage}
\label{\detokenize{BaseImage:labimage}}\label{\detokenize{BaseImage::doc}}\index{LabImage (класс в core)@\spxentry{LabImage}\spxextra{класс в core}}

\begin{fulllineitems}
\phantomsection\label{\detokenize{BaseImage:core.LabImage}}\pysiglinewithargsret{\sphinxbfcode{\sphinxupquote{class }}\sphinxcode{\sphinxupquote{core.}}\sphinxbfcode{\sphinxupquote{LabImage}}}{\emph{path=None}, \emph{image=None}, \emph{pilImage=None}}{}
Базовый класс для работы с изображением.
\begin{description}
\item[{Может быть инициализирован следующими способами:}] \leavevmode\begin{itemize}
\item {} 
передачей параметра \sphinxstylestrong{path}

\item {} 
передачей существующего экземпляра класса {\hyperref[\detokenize{BaseImage:core.LabImage}]{\sphinxcrossref{\sphinxcode{\sphinxupquote{LabImage}}}}}

\item {} 
передачей существующего экземпляра класса \sphinxcode{\sphinxupquote{Image}}

\item {} 
иниализация пустыми параметрами с дальнейшим вызовом функции {\hyperref[\detokenize{BaseImage:core.LabImage.read}]{\sphinxcrossref{\sphinxcode{\sphinxupquote{read()}}}}}

\end{itemize}

\end{description}
\index{calc\_grayscale\_matrix() (метод core.LabImage)@\spxentry{calc\_grayscale\_matrix()}\spxextra{метод core.LabImage}}

\begin{fulllineitems}
\phantomsection\label{\detokenize{BaseImage:core.LabImage.calc_grayscale_matrix}}\pysiglinewithargsret{\sphinxbfcode{\sphinxupquote{calc\_grayscale\_matrix}}}{}{}
Производит расчёт полутоновой матрицы исходного изображения и сохраняет её во внутреннюю переменную

\end{fulllineitems}

\index{read() (метод core.LabImage)@\spxentry{read()}\spxextra{метод core.LabImage}}

\begin{fulllineitems}
\phantomsection\label{\detokenize{BaseImage:core.LabImage.read}}\pysiglinewithargsret{\sphinxbfcode{\sphinxupquote{read}}}{\emph{path: str}}{}
Считывает изображение, расположенное по пути path, и заполняет внутренние переменные класса
\begin{quote}\begin{description}
\item[{Параметры}] \leavevmode
\sphinxstyleliteralstrong{\sphinxupquote{path}} (\sphinxstyleliteralemphasis{\sphinxupquote{str}}) \textendash{} путь до изображения

\end{description}\end{quote}

\end{fulllineitems}

\index{save() (метод core.LabImage)@\spxentry{save()}\spxextra{метод core.LabImage}}

\begin{fulllineitems}
\phantomsection\label{\detokenize{BaseImage:core.LabImage.save}}\pysiglinewithargsret{\sphinxbfcode{\sphinxupquote{save}}}{\emph{name: str}}{}
Сохраняет изображение из внутренней переменной result в виде BMP изображения под заданным в name именем
\begin{quote}\begin{description}
\item[{Параметры}] \leavevmode
\sphinxstyleliteralstrong{\sphinxupquote{name}} (\sphinxstyleliteralemphasis{\sphinxupquote{str}}) \textendash{} имя файла, под которым следует сохранить изображение

\end{description}\end{quote}

\end{fulllineitems}

\index{show() (метод core.LabImage)@\spxentry{show()}\spxextra{метод core.LabImage}}

\begin{fulllineitems}
\phantomsection\label{\detokenize{BaseImage:core.LabImage.show}}\pysiglinewithargsret{\sphinxbfcode{\sphinxupquote{show}}}{}{}
Отображает изображение, сохранённое во внутренней переменной result;
при отсутствии такового отображает исходное изображение

\end{fulllineitems}


\end{fulllineitems}



\section{BinaryImage}
\label{\detokenize{BinaryImage:binaryimage}}\label{\detokenize{BinaryImage::doc}}\index{BinaryImage (класс в BinaryImage)@\spxentry{BinaryImage}\spxextra{класс в BinaryImage}}

\begin{fulllineitems}
\phantomsection\label{\detokenize{BinaryImage:BinaryImage.BinaryImage}}\pysiglinewithargsret{\sphinxbfcode{\sphinxupquote{class }}\sphinxcode{\sphinxupquote{BinaryImage.}}\sphinxbfcode{\sphinxupquote{BinaryImage}}}{\emph{path=None}, \emph{image=None}, \emph{pilImage=None}}{}~\begin{description}
\item[{Класс осуществляющий бинаризацию переданного на вход изображения следующими методами:}] \leavevmode\begin{itemize}
\item {} 
{\hyperref[\detokenize{BinaryImage:BinaryImage.BinaryImage.eikvil_binarisation}]{\sphinxcrossref{\sphinxcode{\sphinxupquote{eikvil\_binarisation()}}}}} \textendash{} метод Эйквила

\item {} 
{\hyperref[\detokenize{BinaryImage:BinaryImage.BinaryImage.cristian_binarisation}]{\sphinxcrossref{\sphinxcode{\sphinxupquote{cristian\_binarisation()}}}}} \textendash{} метод Кристиана

\end{itemize}

\end{description}
\index{calc\_integ() (метод BinaryImage.BinaryImage)@\spxentry{calc\_integ()}\spxextra{метод BinaryImage.BinaryImage}}

\begin{fulllineitems}
\phantomsection\label{\detokenize{BinaryImage:BinaryImage.BinaryImage.calc_integ}}\pysiglinewithargsret{\sphinxbfcode{\sphinxupquote{calc\_integ}}}{\emph{img: numpy.ndarray}}{}
Расчет интегрального изображения из исходного
\begin{quote}\begin{description}
\item[{Параметры}] \leavevmode
\sphinxstyleliteralstrong{\sphinxupquote{img}} (\sphinxstyleliteralemphasis{\sphinxupquote{numpy.ndarray}}) \textendash{} матрица изображения

\end{description}\end{quote}

\end{fulllineitems}

\index{cristian\_binarisation() (метод BinaryImage.BinaryImage)@\spxentry{cristian\_binarisation()}\spxextra{метод BinaryImage.BinaryImage}}

\begin{fulllineitems}
\phantomsection\label{\detokenize{BinaryImage:BinaryImage.BinaryImage.cristian_binarisation}}\pysiglinewithargsret{\sphinxbfcode{\sphinxupquote{cristian\_binarisation}}}{\emph{w\_size=15}, \emph{k=0.5}}{}
Бинаризация изображения методом Кристиана
\begin{quote}\begin{description}
\item[{Параметры}] \leavevmode\begin{itemize}
\item {} 
\sphinxstyleliteralstrong{\sphinxupquote{w\_size}} (\sphinxstyleliteralemphasis{\sphinxupquote{int}}) \textendash{} размер окна

\item {} 
\sphinxstyleliteralstrong{\sphinxupquote{k}} (\sphinxstyleliteralemphasis{\sphinxupquote{float}}) \textendash{} коэффициент, отвечающий за чувствительность бинаризатора

\end{itemize}

\item[{Результат}] \leavevmode
{\hyperref[\detokenize{BaseImage:core.LabImage}]{\sphinxcrossref{\sphinxcode{\sphinxupquote{LabImage}}}}} \textendash{} объект изображения

\end{description}\end{quote}

\end{fulllineitems}

\index{eikvil\_binarisation() (метод BinaryImage.BinaryImage)@\spxentry{eikvil\_binarisation()}\spxextra{метод BinaryImage.BinaryImage}}

\begin{fulllineitems}
\phantomsection\label{\detokenize{BinaryImage:BinaryImage.BinaryImage.eikvil_binarisation}}\pysiglinewithargsret{\sphinxbfcode{\sphinxupquote{eikvil\_binarisation}}}{\emph{rsize=3}, \emph{Rsize=15}, \emph{eps=15}}{}
Бинаризация изображения методом Эйквила
\begin{quote}\begin{description}
\item[{Параметры}] \leavevmode\begin{itemize}
\item {} 
\sphinxstyleliteralstrong{\sphinxupquote{rsize}} (\sphinxstyleliteralemphasis{\sphinxupquote{int}}) \textendash{} размер малого окна

\item {} 
\sphinxstyleliteralstrong{\sphinxupquote{Rsize}} (\sphinxstyleliteralemphasis{\sphinxupquote{int}}) \textendash{} размер большего окна

\item {} 
\sphinxstyleliteralstrong{\sphinxupquote{eps}} (\sphinxstyleliteralemphasis{\sphinxupquote{int}}) \textendash{} величина отклонения для математических ожиданий чёрного и белого, в пределах которого можно считать         , что они отличются несущественно

\end{itemize}

\item[{Результат}] \leavevmode
{\hyperref[\detokenize{BaseImage:core.LabImage}]{\sphinxcrossref{\sphinxcode{\sphinxupquote{LabImage}}}}} \textendash{} объект изображения

\item[{Raises}] \leavevmode
WrongWindowSize

\end{description}\end{quote}

\end{fulllineitems}


\end{fulllineitems}



\section{FilteredImage}
\label{\detokenize{FilteredImage:filteredimage}}\label{\detokenize{FilteredImage::doc}}\index{FilteredImage (класс в FilteredImage)@\spxentry{FilteredImage}\spxextra{класс в FilteredImage}}

\begin{fulllineitems}
\phantomsection\label{\detokenize{FilteredImage:FilteredImage.FilteredImage}}\pysiglinewithargsret{\sphinxbfcode{\sphinxupquote{class }}\sphinxcode{\sphinxupquote{FilteredImage.}}\sphinxbfcode{\sphinxupquote{FilteredImage}}}{\emph{path=None}, \emph{image=None}}{}~\begin{description}
\item[{Класс осуществляющий фильтрацию переданного на вход изображения следующими методами:}] \leavevmode\begin{itemize}
\item {} 
{\hyperref[\detokenize{FilteredImage:FilteredImage.FilteredImage.median_filter}]{\sphinxcrossref{\sphinxcode{\sphinxupquote{median\_filter()}}}}} \textendash{} медианная фильтрация

\item {} 
{\hyperref[\detokenize{FilteredImage:FilteredImage.FilteredImage.rank_filter}]{\sphinxcrossref{\sphinxcode{\sphinxupquote{rank\_filter()}}}}} \textendash{} ранговая фильтрация

\item {} 
{\hyperref[\detokenize{FilteredImage:FilteredImage.FilteredImage.weighted_rank_filter}]{\sphinxcrossref{\sphinxcode{\sphinxupquote{weighted\_rank\_filter()}}}}} \textendash{} взвешанная ранговая фильтрация

\end{itemize}

\end{description}
\index{median\_filter() (метод FilteredImage.FilteredImage)@\spxentry{median\_filter()}\spxextra{метод FilteredImage.FilteredImage}}

\begin{fulllineitems}
\phantomsection\label{\detokenize{FilteredImage:FilteredImage.FilteredImage.median_filter}}\pysiglinewithargsret{\sphinxbfcode{\sphinxupquote{median\_filter}}}{\emph{wsize=3}}{}
Медианная фильтрация изображения
\begin{quote}\begin{description}
\item[{Параметры}] \leavevmode
\sphinxstyleliteralstrong{\sphinxupquote{wsize}} (\sphinxstyleliteralemphasis{\sphinxupquote{int}}) \textendash{} размер окна фильтрации

\item[{Результат}] \leavevmode
{\hyperref[\detokenize{BaseImage:core.LabImage}]{\sphinxcrossref{\sphinxcode{\sphinxupquote{LabImage}}}}} \textendash{} объект изображения

\item[{Raises}] \leavevmode
WrongWindowSize

\end{description}\end{quote}

\end{fulllineitems}

\index{rank\_filter() (метод FilteredImage.FilteredImage)@\spxentry{rank\_filter()}\spxextra{метод FilteredImage.FilteredImage}}

\begin{fulllineitems}
\phantomsection\label{\detokenize{FilteredImage:FilteredImage.FilteredImage.rank_filter}}\pysiglinewithargsret{\sphinxbfcode{\sphinxupquote{rank\_filter}}}{\emph{rank: int}, \emph{wsize=3}}{}
Невзвешенная ранговая фильтрация изображения
\begin{quote}\begin{description}
\item[{Параметры}] \leavevmode\begin{itemize}
\item {} 
\sphinxstyleliteralstrong{\sphinxupquote{rank}} (\sphinxstyleliteralemphasis{\sphinxupquote{int}}) \textendash{} ранг фильтра

\item {} 
\sphinxstyleliteralstrong{\sphinxupquote{wsize}} (\sphinxstyleliteralemphasis{\sphinxupquote{int}}) \textendash{} размер окна фильтрации (поддерживаются только окна размера 3 или 5)

\end{itemize}

\item[{Результат}] \leavevmode
{\hyperref[\detokenize{BaseImage:core.LabImage}]{\sphinxcrossref{\sphinxcode{\sphinxupquote{LabImage}}}}} \textendash{} объект изображения

\item[{Raises}] \leavevmode
WrongWindowSize, WrongRank

\end{description}\end{quote}

\end{fulllineitems}

\index{weighted\_rank\_filter() (метод FilteredImage.FilteredImage)@\spxentry{weighted\_rank\_filter()}\spxextra{метод FilteredImage.FilteredImage}}

\begin{fulllineitems}
\phantomsection\label{\detokenize{FilteredImage:FilteredImage.FilteredImage.weighted_rank_filter}}\pysiglinewithargsret{\sphinxbfcode{\sphinxupquote{weighted\_rank\_filter}}}{\emph{rank: int}, \emph{wsize=3}}{}
Взвешенная ранговая фильтрация изображения
\begin{quote}\begin{description}
\item[{Параметры}] \leavevmode\begin{itemize}
\item {} 
\sphinxstyleliteralstrong{\sphinxupquote{rank}} (\sphinxstyleliteralemphasis{\sphinxupquote{int}}) \textendash{} ранг фильтра

\item {} 
\sphinxstyleliteralstrong{\sphinxupquote{wsize}} (\sphinxstyleliteralemphasis{\sphinxupquote{int}}) \textendash{} размер окна фильтрации (поддерживаются только окна размера 3 или 5)

\end{itemize}

\item[{Результат}] \leavevmode
{\hyperref[\detokenize{BaseImage:core.LabImage}]{\sphinxcrossref{\sphinxcode{\sphinxupquote{LabImage}}}}} \textendash{} объект изображения

\item[{Raises}] \leavevmode
WrongWindowSize

\end{description}\end{quote}

\end{fulllineitems}


\end{fulllineitems}



\section{ContouredImage}
\label{\detokenize{ContouredImage:contouredimage}}\label{\detokenize{ContouredImage::doc}}\index{ContouredImage (класс в ContouredImage)@\spxentry{ContouredImage}\spxextra{класс в ContouredImage}}

\begin{fulllineitems}
\phantomsection\label{\detokenize{ContouredImage:ContouredImage.ContouredImage}}\pysiglinewithargsret{\sphinxbfcode{\sphinxupquote{class }}\sphinxcode{\sphinxupquote{ContouredImage.}}\sphinxbfcode{\sphinxupquote{ContouredImage}}}{\emph{path=None}, \emph{image=None}}{}~\begin{description}
\item[{Класс осуществляющий выделение контуров переданного на вход изображения следующими методами:}] \leavevmode\begin{itemize}
\item {} 
{\hyperref[\detokenize{ContouredImage:ContouredImage.ContouredImage.prewitt_operator}]{\sphinxcrossref{\sphinxcode{\sphinxupquote{prewitt\_operator()}}}}} \textendash{} контурирование оператором Пюитт

\item {} 
{\hyperref[\detokenize{ContouredImage:ContouredImage.ContouredImage.sobel_operator}]{\sphinxcrossref{\sphinxcode{\sphinxupquote{sobel\_operator()}}}}} \textendash{} контурирование оператором Собеля

\end{itemize}

\end{description}
\index{prewitt\_operator() (метод ContouredImage.ContouredImage)@\spxentry{prewitt\_operator()}\spxextra{метод ContouredImage.ContouredImage}}

\begin{fulllineitems}
\phantomsection\label{\detokenize{ContouredImage:ContouredImage.ContouredImage.prewitt_operator}}\pysiglinewithargsret{\sphinxbfcode{\sphinxupquote{prewitt\_operator}}}{}{}
Контурирование оператором Собеля
\begin{quote}\begin{description}
\item[{Результат}] \leavevmode
{\hyperref[\detokenize{BaseImage:core.LabImage}]{\sphinxcrossref{\sphinxcode{\sphinxupquote{LabImage}}}}} \textendash{} объект изображения

\end{description}\end{quote}

\end{fulllineitems}

\index{sobel\_operator() (метод ContouredImage.ContouredImage)@\spxentry{sobel\_operator()}\spxextra{метод ContouredImage.ContouredImage}}

\begin{fulllineitems}
\phantomsection\label{\detokenize{ContouredImage:ContouredImage.ContouredImage.sobel_operator}}\pysiglinewithargsret{\sphinxbfcode{\sphinxupquote{sobel\_operator}}}{\emph{t}}{}
Контурирование оператором Собеля
\begin{quote}\begin{description}
\item[{Параметры}] \leavevmode
\sphinxstyleliteralstrong{\sphinxupquote{t}} (\sphinxstyleliteralemphasis{\sphinxupquote{int}}) \textendash{} порог

\item[{Результат}] \leavevmode
{\hyperref[\detokenize{BaseImage:core.LabImage}]{\sphinxcrossref{\sphinxcode{\sphinxupquote{LabImage}}}}} \textendash{} объект изображения

\end{description}\end{quote}

\end{fulllineitems}


\end{fulllineitems}



\section{SymbolImage}
\label{\detokenize{SymbolImage:symbolimage}}\label{\detokenize{SymbolImage::doc}}\index{SymbolImage (класс в SymbolImage)@\spxentry{SymbolImage}\spxextra{класс в SymbolImage}}

\begin{fulllineitems}
\phantomsection\label{\detokenize{SymbolImage:SymbolImage.SymbolImage}}\pysiglinewithargsret{\sphinxbfcode{\sphinxupquote{class }}\sphinxcode{\sphinxupquote{SymbolImage.}}\sphinxbfcode{\sphinxupquote{SymbolImage}}}{\emph{path=None}, \emph{image=None}}{}
Класс осуществляющий выделение символьных признаков для заданного изображения
\index{calc\_characteristics() (метод SymbolImage.SymbolImage)@\spxentry{calc\_characteristics()}\spxextra{метод SymbolImage.SymbolImage}}

\begin{fulllineitems}
\phantomsection\label{\detokenize{SymbolImage:SymbolImage.SymbolImage.calc_characteristics}}\pysiglinewithargsret{\sphinxbfcode{\sphinxupquote{calc\_characteristics}}}{}{}
Функция вычисления характеристик букв алфавита
\begin{quote}\begin{description}
\item[{Результат}] \leavevmode
dict \textendash{} характеристики символа

\end{description}\end{quote}

\end{fulllineitems}


\end{fulllineitems}



\section{FontCharacteristics}
\label{\detokenize{FontCharacteristics:fontcharacteristics}}\label{\detokenize{FontCharacteristics::doc}}\index{FontCharacteristics (класс в SymbolImage)@\spxentry{FontCharacteristics}\spxextra{класс в SymbolImage}}

\begin{fulllineitems}
\phantomsection\label{\detokenize{FontCharacteristics:SymbolImage.FontCharacteristics}}\pysiglinewithargsret{\sphinxbfcode{\sphinxupquote{class }}\sphinxcode{\sphinxupquote{SymbolImage.}}\sphinxbfcode{\sphinxupquote{FontCharacteristics}}}{\emph{symbols: list}, \emph{font=None}, \emph{font\_size=None}, \emph{symbol\_size=None}}{}
Класс создающий набор признаков для букв заданного алфавита
\index{calc\_characteristics() (метод SymbolImage.FontCharacteristics)@\spxentry{calc\_characteristics()}\spxextra{метод SymbolImage.FontCharacteristics}}

\begin{fulllineitems}
\phantomsection\label{\detokenize{FontCharacteristics:SymbolImage.FontCharacteristics.calc_characteristics}}\pysiglinewithargsret{\sphinxbfcode{\sphinxupquote{calc\_characteristics}}}{}{}
Функция характеристик сиволов алфавита
:return: LabImage \textendash{} объект изображения

\end{fulllineitems}

\index{create\_symbol\_images() (метод SymbolImage.FontCharacteristics)@\spxentry{create\_symbol\_images()}\spxextra{метод SymbolImage.FontCharacteristics}}

\begin{fulllineitems}
\phantomsection\label{\detokenize{FontCharacteristics:SymbolImage.FontCharacteristics.create_symbol_images}}\pysiglinewithargsret{\sphinxbfcode{\sphinxupquote{create\_symbol\_images}}}{}{{ $\rightarrow$ None}}
Функция генерации изображений сиволов алфавита

\end{fulllineitems}

\index{to\_csv() (метод SymbolImage.FontCharacteristics)@\spxentry{to\_csv()}\spxextra{метод SymbolImage.FontCharacteristics}}

\begin{fulllineitems}
\phantomsection\label{\detokenize{FontCharacteristics:SymbolImage.FontCharacteristics.to_csv}}\pysiglinewithargsret{\sphinxbfcode{\sphinxupquote{to\_csv}}}{\emph{name: str}}{}
Создание файла с характеристиками символов алфавита .csv
\begin{quote}\begin{description}
\item[{Параметры}] \leavevmode
\sphinxstyleliteralstrong{\sphinxupquote{name}} (\sphinxstyleliteralemphasis{\sphinxupquote{str}}\sphinxstyleliteralemphasis{\sphinxupquote{ or }}\sphinxstyleliteralemphasis{\sphinxupquote{None}}) \textendash{} путь до файла csv

\end{description}\end{quote}

\end{fulllineitems}


\end{fulllineitems}



\section{TextProfiler}
\label{\detokenize{TextProfiler:textprofiler}}\label{\detokenize{TextProfiler::doc}}\index{TextProfiler (класс в TextProfiler)@\spxentry{TextProfiler}\spxextra{класс в TextProfiler}}

\begin{fulllineitems}
\phantomsection\label{\detokenize{TextProfiler:TextProfiler.TextProfiler}}\pysiglinewithargsret{\sphinxbfcode{\sphinxupquote{class }}\sphinxcode{\sphinxupquote{TextProfiler.}}\sphinxbfcode{\sphinxupquote{TextProfiler}}}{\emph{image=None}, \emph{path=None}}{}
Класс осуществляющий выделение букв в тексте
\index{get\_text\_segmentation() (метод TextProfiler.TextProfiler)@\spxentry{get\_text\_segmentation()}\spxextra{метод TextProfiler.TextProfiler}}

\begin{fulllineitems}
\phantomsection\label{\detokenize{TextProfiler:TextProfiler.TextProfiler.get_text_segmentation}}\pysiglinewithargsret{\sphinxbfcode{\sphinxupquote{get\_text\_segmentation}}}{\emph{t=0}}{}
Получение координат символов на изображении
\begin{quote}\begin{description}
\item[{Параметры}] \leavevmode
\sphinxstyleliteralstrong{\sphinxupquote{t}} (\sphinxstyleliteralemphasis{\sphinxupquote{int}}) \textendash{} порог

\item[{Результат}] \leavevmode
{\hyperref[\detokenize{BaseImage:core.LabImage}]{\sphinxcrossref{\sphinxcode{\sphinxupquote{LabImage}}}}} \textendash{} объект изображения

\end{description}\end{quote}

\end{fulllineitems}


\end{fulllineitems}

\phantomsection\label{\detokenize{TextProfiler:module-TextProfiler}}\index{TextProfiler (модуль)@\spxentry{TextProfiler}\spxextra{модуль}}\index{get\_y\_profile() (в модуле TextProfiler)@\spxentry{get\_y\_profile()}\spxextra{в модуле TextProfiler}}

\begin{fulllineitems}
\phantomsection\label{\detokenize{TextProfiler:TextProfiler.get_y_profile}}\pysiglinewithargsret{\sphinxcode{\sphinxupquote{TextProfiler.}}\sphinxbfcode{\sphinxupquote{get\_y\_profile}}}{\emph{img}}{}
Получение вертикального профиля изображения
\begin{quote}\begin{description}
\item[{Параметры}] \leavevmode
\sphinxstyleliteralstrong{\sphinxupquote{img}} (\sphinxcode{\sphinxupquote{Image}}) \textendash{} изображение

\item[{Результат}] \leavevmode
\sphinxcode{\sphinxupquote{ndarray}} \textendash{} массив с профилем изображения

\end{description}\end{quote}

\end{fulllineitems}

\index{get\_x\_profile() (в модуле TextProfiler)@\spxentry{get\_x\_profile()}\spxextra{в модуле TextProfiler}}

\begin{fulllineitems}
\phantomsection\label{\detokenize{TextProfiler:TextProfiler.get_x_profile}}\pysiglinewithargsret{\sphinxcode{\sphinxupquote{TextProfiler.}}\sphinxbfcode{\sphinxupquote{get\_x\_profile}}}{\emph{img}}{}
Получение горизонтального профиля изображения
\begin{quote}\begin{description}
\item[{Параметры}] \leavevmode
\sphinxstyleliteralstrong{\sphinxupquote{img}} (\sphinxcode{\sphinxupquote{Image}}) \textendash{} изображение

\item[{Результат}] \leavevmode
\sphinxcode{\sphinxupquote{ndarray}} \textendash{} массив с профилем изображения

\end{description}\end{quote}

\end{fulllineitems}

\index{find\_zero() (в модуле TextProfiler)@\spxentry{find\_zero()}\spxextra{в модуле TextProfiler}}

\begin{fulllineitems}
\phantomsection\label{\detokenize{TextProfiler:TextProfiler.find_zero}}\pysiglinewithargsret{\sphinxcode{\sphinxupquote{TextProfiler.}}\sphinxbfcode{\sphinxupquote{find\_zero}}}{\emph{arr}, \emph{t}}{}
Подсчет нулей в профиле изображения
\begin{quote}\begin{description}
\item[{Параметры}] \leavevmode\begin{itemize}
\item {} 
\sphinxstyleliteralstrong{\sphinxupquote{arr}} (\sphinxcode{\sphinxupquote{ndarray}}) \textendash{} профиль

\item {} 
\sphinxstyleliteralstrong{\sphinxupquote{t}} (\sphinxstyleliteralemphasis{\sphinxupquote{int}}) \textendash{} порог

\end{itemize}

\item[{Результат}] \leavevmode
int \textendash{} количество нулей

\end{description}\end{quote}

\end{fulllineitems}

\index{get\_zones() (в модуле TextProfiler)@\spxentry{get\_zones()}\spxextra{в модуле TextProfiler}}

\begin{fulllineitems}
\phantomsection\label{\detokenize{TextProfiler:TextProfiler.get_zones}}\pysiglinewithargsret{\sphinxcode{\sphinxupquote{TextProfiler.}}\sphinxbfcode{\sphinxupquote{get\_zones}}}{\emph{prof}, \emph{r}, \emph{t}}{}
Опрделение координат зон текста: для вертикального профиля \sphinxhyphen{} строки, для горизонтального \sphinxhyphen{} буквы
\begin{quote}\begin{description}
\item[{Параметры}] \leavevmode\begin{itemize}
\item {} 
\sphinxstyleliteralstrong{\sphinxupquote{prof}} (\sphinxcode{\sphinxupquote{ndarray}}) \textendash{} профиль

\item {} 
\sphinxstyleliteralstrong{\sphinxupquote{t}} (\sphinxstyleliteralemphasis{\sphinxupquote{int}}) \textendash{} порог

\item {} 
\sphinxstyleliteralstrong{\sphinxupquote{r}} (\sphinxstyleliteralemphasis{\sphinxupquote{размер окна}}) \textendash{} размер окна

\end{itemize}

\item[{Результат}] \leavevmode
int \textendash{} количество нулей

\end{description}\end{quote}

\end{fulllineitems}

\index{get\_letters\_in\_row() (в модуле TextProfiler)@\spxentry{get\_letters\_in\_row()}\spxextra{в модуле TextProfiler}}

\begin{fulllineitems}
\phantomsection\label{\detokenize{TextProfiler:TextProfiler.get_letters_in_row}}\pysiglinewithargsret{\sphinxcode{\sphinxupquote{TextProfiler.}}\sphinxbfcode{\sphinxupquote{get\_letters\_in\_row}}}{\emph{prof}, \emph{y\_start}, \emph{y\_end}, \emph{t}}{}
Опрделение координат букв
\begin{quote}\begin{description}
\item[{Параметры}] \leavevmode\begin{itemize}
\item {} 
\sphinxstyleliteralstrong{\sphinxupquote{prof}} (\sphinxcode{\sphinxupquote{ndarray}}) \textendash{} профиль

\item {} 
\sphinxstyleliteralstrong{\sphinxupquote{y\_start}} (\sphinxstyleliteralemphasis{\sphinxupquote{int}}) \textendash{} координаты начала строки

\item {} 
\sphinxstyleliteralstrong{\sphinxupquote{y\_end}} (\sphinxstyleliteralemphasis{\sphinxupquote{int}}) \textendash{} координаты конца строки

\item {} 
\sphinxstyleliteralstrong{\sphinxupquote{t}} (\sphinxstyleliteralemphasis{\sphinxupquote{int}}) \textendash{} порог

\end{itemize}

\item[{Результат}] \leavevmode
\sphinxcode{\sphinxupquote{ndarray}} \textendash{} координаты букв

\end{description}\end{quote}

\end{fulllineitems}

\index{get\_rows\_in\_text() (в модуле TextProfiler)@\spxentry{get\_rows\_in\_text()}\spxextra{в модуле TextProfiler}}

\begin{fulllineitems}
\phantomsection\label{\detokenize{TextProfiler:TextProfiler.get_rows_in_text}}\pysiglinewithargsret{\sphinxcode{\sphinxupquote{TextProfiler.}}\sphinxbfcode{\sphinxupquote{get\_rows\_in\_text}}}{\emph{prof}, \emph{t}}{}
Опрделение координат строк
\begin{quote}\begin{description}
\item[{Параметры}] \leavevmode\begin{itemize}
\item {} 
\sphinxstyleliteralstrong{\sphinxupquote{prof}} (\sphinxcode{\sphinxupquote{ndarray}}) \textendash{} горизональный профиль

\item {} 
\sphinxstyleliteralstrong{\sphinxupquote{t}} (\sphinxstyleliteralemphasis{\sphinxupquote{int}}) \textendash{} порог

\end{itemize}

\item[{Результат}] \leavevmode
\sphinxcode{\sphinxupquote{ndarray}} \textendash{} координаты строк

\end{description}\end{quote}

\end{fulllineitems}

\index{draw\_segmented\_row() (в модуле TextProfiler)@\spxentry{draw\_segmented\_row()}\spxextra{в модуле TextProfiler}}

\begin{fulllineitems}
\phantomsection\label{\detokenize{TextProfiler:TextProfiler.draw_segmented_row}}\pysiglinewithargsret{\sphinxcode{\sphinxupquote{TextProfiler.}}\sphinxbfcode{\sphinxupquote{draw\_segmented\_row}}}{\emph{img}, \emph{zones}}{}
Отрисовка сегментации текста на буквы
\begin{quote}\begin{description}
\item[{Параметры}] \leavevmode\begin{itemize}
\item {} 
\sphinxstyleliteralstrong{\sphinxupquote{img}} (\sphinxcode{\sphinxupquote{Image}}) \textendash{} изображение

\item {} 
\sphinxstyleliteralstrong{\sphinxupquote{zones}} (\sphinxcode{\sphinxupquote{ndarray}}) \textendash{} координаты букв

\end{itemize}

\item[{Результат}] \leavevmode
\sphinxcode{\sphinxupquote{Image}} \sphinxhyphen{} размеченное изображение

\end{description}\end{quote}

\end{fulllineitems}



\section{CharsRecognizer}
\label{\detokenize{CharsRecognizer:charsrecognizer}}\label{\detokenize{CharsRecognizer::doc}}\index{CharsRecognizer (класс в CharsRecognizer)@\spxentry{CharsRecognizer}\spxextra{класс в CharsRecognizer}}

\begin{fulllineitems}
\phantomsection\label{\detokenize{CharsRecognizer:CharsRecognizer.CharsRecognizer}}\pysiglinewithargsret{\sphinxbfcode{\sphinxupquote{class }}\sphinxcode{\sphinxupquote{CharsRecognizer.}}\sphinxbfcode{\sphinxupquote{CharsRecognizer}}}{\emph{path=None}, \emph{image=None}, \emph{font=\textquotesingle{}TNR.ttf\textquotesingle{}}, \emph{font\_size=36}}{}
Класс распознования символов на изображении с указанием параметров шрифта
\index{tryToRecognizeWithFont() (метод CharsRecognizer.CharsRecognizer)@\spxentry{tryToRecognizeWithFont()}\spxextra{метод CharsRecognizer.CharsRecognizer}}

\begin{fulllineitems}
\phantomsection\label{\detokenize{CharsRecognizer:CharsRecognizer.CharsRecognizer.tryToRecognizeWithFont}}\pysiglinewithargsret{\sphinxbfcode{\sphinxupquote{tryToRecognizeWithFont}}}{\emph{font=None}, \emph{fontSize=None}, \emph{symbol\_size=(50}, \emph{50)}}{}
Функция разпознавания символов в тексте
\begin{quote}\begin{description}
\item[{Параметры}] \leavevmode\begin{itemize}
\item {} 
\sphinxstyleliteralstrong{\sphinxupquote{path}} (\sphinxstyleliteralemphasis{\sphinxupquote{str}}\sphinxstyleliteralemphasis{\sphinxupquote{ or }}\sphinxstyleliteralemphasis{\sphinxupquote{None}}) \textendash{} путь до изображения

\item {} 
\sphinxstyleliteralstrong{\sphinxupquote{image}} ({\hyperref[\detokenize{BaseImage:core.LabImage}]{\sphinxcrossref{\sphinxcode{\sphinxupquote{LabImage}}}}} or None) \textendash{} экземпляр класса {\hyperref[\detokenize{BaseImage:core.LabImage}]{\sphinxcrossref{\sphinxcode{\sphinxupquote{LabImage}}}}}

\item {} 
\sphinxstyleliteralstrong{\sphinxupquote{font}} (\sphinxstyleliteralemphasis{\sphinxupquote{str}}\sphinxstyleliteralemphasis{\sphinxupquote{ or }}\sphinxstyleliteralemphasis{\sphinxupquote{None}}) \textendash{} путь до файла шрифта

\item {} 
\sphinxstyleliteralstrong{\sphinxupquote{fontSize}} (\sphinxstyleliteralemphasis{\sphinxupquote{int}}) \textendash{} размер шрифта

\item {} 
\sphinxstyleliteralstrong{\sphinxupquote{symbol\_size}} (\sphinxstyleliteralemphasis{\sphinxupquote{tuple}}) \textendash{} размер символа

\end{itemize}

\item[{Результат}] \leavevmode
str \textendash{} распознаные символы

\end{description}\end{quote}

\end{fulllineitems}


\end{fulllineitems}



\chapter{Indices and tables}
\label{\detokenize{index:indices-and-tables}}\begin{itemize}
\item {} 
\DUrole{xref,std,std-ref}{genindex}

\item {} 
\DUrole{xref,std,std-ref}{search}

\end{itemize}


\renewcommand{\indexname}{Содержание модулей Python}
\begin{sphinxtheindex}
\let\bigletter\sphinxstyleindexlettergroup
\bigletter{t}
\item\relax\sphinxstyleindexentry{TextProfiler}\sphinxstyleindexpageref{TextProfiler:\detokenize{module-TextProfiler}}
\end{sphinxtheindex}

\renewcommand{\indexname}{Алфавитный указатель}
\printindex
\end{document}